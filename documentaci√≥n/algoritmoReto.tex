\documentclass[a4paper,10pt]{scrartcl}
\usepackage[utf8]{inputenc}
\usepackage{verbatim}
\usepackage{algorithm}
\usepackage{algorithmic}
\usepackage{amsmath}
\usepackage[usenames,dvipsnames,svgnames,table]{xcolor}

\everymath{\displaystyle}
\floatname{algorithm}{Algoritmo}
\renewcommand{\listalgorithmname}{Lista de algoritmos}
\renewcommand{\algorithmicrequire}{\textbf{Entrada:}}
\renewcommand{\algorithmicensure}{\textbf{Salida:}}
\renewcommand{\algorithmicend}{\textbf{Fin}}
\renewcommand{\algorithmicif}{\textbf{Si}}
\renewcommand{\algorithmicthen}{\textbf{entonces}}
\renewcommand{\algorithmicelse}{\textbf{Si no}}
\renewcommand{\algorithmicelsif}{\algorithmicelse,\ \algorithmicif}
\renewcommand{\algorithmicendif}{\algorithmicend\ \algorithmicif}
\renewcommand{\algorithmicfor}{\textbf{Para }}
\renewcommand{\algorithmicforall}{\textbf{para todo}}
\renewcommand{\algorithmicdo}{\textbf{}}
\renewcommand{\algorithmicendfor}{\algorithmicend\ \algorithmicfor}
\renewcommand{\algorithmicwhile}{\textbf{mientras}}
\renewcommand{\algorithmicendwhile}{\algorithmicend\ \algorithmicwhile}
\renewcommand{\algorithmicloop}{\textbf{repetir}}
\renewcommand{\algorithmicendloop}{\algorithmicend\ \algorithmicloop}
\renewcommand{\algorithmicrepeat}{\textbf{repetir}}
\renewcommand{\algorithmicuntil}{\textbf{hasta que}}
\renewcommand{\algorithmicprint}{\textbf{imprimir}} 
\renewcommand{\algorithmicreturn}{\textbf{devolver}} 
\renewcommand{\algorithmictrue}{\textbf{true }} 
\renewcommand{\algorithmicfalse}{\textbf{false }} 
  % Variables de entorno en español
\setlength{\parindent}{0cm}

\newcommand{\objeto}[4]{\indent\underline{\textbf{TDA #1}}
            \begin{center}\begin{tabular}{|p{7cm}|}
                 \hline
                 \multicolumn{1}{|c|}{\texttt{#1}}\\
                 \hline
                  \\#2\\\\
                 \hline
                 \\#3\\\\
                 \hline
            \end{tabular}\end{center}
            \begin{itemize}
            #4
            \end{itemize}
               }
\def\C++#1{\texttt{#1}}
\def\T_{\texttt{T}}
%opening
\title{Reto II: Problema de las cifras}
\author{Francisco David Charte Luque \and 
        Ignacio Cordón Castillo \and
        Mario Román García}
\date{}

\renewcommand*\contentsname{Contenidos}
\begin{document}
\maketitle

\tableofcontents

\newpage
\section{Tipos de datos abstractos usados.}

Se emplea la siguiente notación genérica para la representación 
de un TDA abstracto:\\

\objeto{MiTDA}{
		\C++{- datos privados}\\
		\C++{- métodos privados}
	}{
		\C++{+ datos accesibles a través de la interfaz}\\
		\C++{+ métodos invocables desde la interfaz}
	}{\item Descripciones sobre el TDA}

\bigskip
Los TDA empleados en la resolución del problema de las cifras han sido:\\

\objeto{Cuenta}{}{
        \C++{+ primero}\\
        \C++{+ segundo}\\
        \C++{+ operador}\\
        \C++{+ resultado}
    }{
        \item \C++{primero} Número entero, representando el primer operando.
        \item \C++{segundo} Número entero, que representa el segundo operando.
        \item \C++{operador} Carácter que corresponde a la operación realizada sobre los números.
        \item \C++{resultado} Número entero, resultado de realizar la operación sobre \C++{primero} y \C++{segundo}.
    }
 %\newpage 
 \objeto{ProblemaCifras}{\C++{- numerosIniciales}\\
                        \C++{- numeros}\\
                        \C++{- operaciones}\\
                        \C++{- meta}\\
                        \C++{- operacionesPosibles}\\
                        }
                        {\C++{+ opera()}\\
                        \C++{+ resuelve()}}{
  \item \C++{numerosIniciales} Conjunto que almacena los enteros
   a partir de los que se pretende obtener \C++{meta}
  \item \C++{numeros} Lista sobre la que se realizarán todas las
   operaciones necesarias hasta llegar a una aproximación (o al número
   buscado exactamente) de \C++{meta}
  \item \C++{operaciones} Lista de objetos \C++{Cuenta} en los que
   se almacenarán las operaciones realizadas hasta llegar a \C++{meta},
   o a una aproximación a \C++{meta}
  \item \C++{meta} Entero positivo de 3 cifras a aproximar, y en
   caso de ser posible, hallar de forma exacta mediante operaciones sobre
   las cifras dadas iniciales
  \item \C++{operacionesPosibles} Conjunto que contiene todas las
   operaciones posibles aplicables \{\C++{+,*,-,/}\}
  \item \C++{opera()} Función que devuelve para dos operandos dados, el resultado
   de una operación determinada de entre \C++{operacionesPosibles} para ellos
  \item \C++{resuelve()} Función recursiva que selecciona parejas de cifras
   de \C++{numeros}, que introduce operados (con \C++{opera()}) en dicha lista, 
   para llamarse a sí misma e intentar llegar a \C++{meta}. Caso de no producir
   acierto, saca los números introducidos y devuelve los extraídos, y reitera con 
   otra pareja
  }
  
  \newpage
  \objeto{Números}{\C++{- contenedorNumeros}\\
  }{
		   \C++{+ insertarResultado(resultado)}\\
		   \C++{+ retirarNumero(posición)}\\
  }{
    \item \C++{contenedorNumeros} Contenedor en el que se almacenarán los números,
    dependiendo de la implementación, podría ser un vector, una cola, una doble cola o un vector circular.
    \item \C++{insertarResultado} Inserta un nuevo resultado en el contenedor de números,
    con el que se podrá operar posteriormente.
    \item \C++{retirarNumero} Retira un número ya usado, que ya no se podrá usar en siguientes operaciones.
  }
  
  \objeto{Operación}{
  }{
		   \C++{+ funcionEnteros(int, int)}\\
		   \C++{+ esPosible(int, int)}\\
  }{
    \item \C++{funcionEntreDosEnteros} Función $f: \C++{int} \times \C++{int} \rightarrow \C++{int}$,
    en forma de función anónima lambda que representará una de las operaciones posibles entre dos
    números del conjunto. En nuestro caso, sumas, restas, multiplicaciones y divisiones.
    \item \C++{esPosible} Verdadero si se cumplen las condiciones que permiten que la operación
    produzca un entero positivo.
  }
   % ¿Faltaría el TDA Números?
   % ¿Faltaría el TDA Operación?
   
   \newpage
   % Recreo el algoritmo intentando usar el más alto nivel posible.
   
\section{Algoritmos.}
\subsection{Algoritmo principal.}
   El algoritmo propuesto para resolver el problema de las cifras es:\\
   
   \begin{algorithm}[H]
   \begin{algorithmic}[1]
   \REQUIRE \ \\
	    \C++{meta}, número a aproximar\\
	    \C++{numeros}, enteros aleatorios iniciales del conjunto\\
            \C++{size}, número de posiciones de la lista \C++{números}    
            
   \ENSURE \ \\
	   \TRUE {si logramos alcanzar exactamente \C++{meta} o es una
			 de las cifras de \C++{numeros}}\\
	   \FALSE {si sólo logramos una aproximación \C++{aprox} a \C++{meta}}
   
	\STATE { Inicializa \C++{mejor\_aprox} a -1. } 
	\IF{ Hay al menos dos cifras que seleccionar}
		\STATE{
		\FOR{cada pareja ordenada \C++{(a,b)} en \C++{numeros}}
			\STATE{ 
			\FOR{cada operación \C++{op} en \C++{[+,*,-,/]}} 
			\IF { \C++{a (op) b} es posible }
				\STATE { Computa la \C++{cuenta} }
				\STATE { Almacena la cuenta en la pila de cuentas }
				\STATE { Retira \C++{a,b} del conjunto de números }
				\STATE { Introduce \C++{a (op) b} en el conjunto de números }
				\IF { $|\C++{meta - a (op) b}| < |\C++{meta - mejor\_aprox}|$  }
				\STATE {$\C++{mejor\_aprox := meta}$}
				\IF {$\C++{mejor\_aprox == meta}$}
				\RETURN \TRUE
				\ENDIF
				\ENDIF
				\IF { llamamos recursivamente al algoritmo sobre \C++{números} y devuelve \TRUE }
				\RETURN \TRUE
				\ENDIF
				\STATE { Retira la \C++{cuenta} de la pila de cuentas }
				\STATE { Retira \C++{a (op) b} del conjunto de números }
				\STATE { Reintroduce \C++{a,b} en el conjunto de números }
			\ENDIF
			\ENDFOR
			}
		\ENDFOR
		}
    \ELSE
		\RETURN\FALSE
	\ENDIF
   \end{algorithmic}
   \caption{ALGORITMO DE CÁLCULO DEL NÚMERO DE 3 CIFRAS}
   \label{Cifras}
   \end{algorithm}
   
   Llamando $T(n)$ a la función que da la eficiencia del algoritmo \ref{Cifras},
   en función de la longitud del vector \C++{numeros}, esto es, de las cifras
   dadas para llegar a meta, se tiene:
   \begin{itemize} 
    \item [-] Desde las líneas 1 a 3, las operaciones realizadas son $\mathcal{O}(1)$
    \item [-] La selección de parejas \C++{(a,b)} de la línea 4 se hace mediante
    combinaciones ${a \choose b}$. Explícitamente, podemos observar como en
    la implementación en \C++{C++} adjunta, esto supone:
    
    \begin{gather*}
    \begin{split}
        &\sum_{i=0}^{n-1}(n-i)=\sum_{i=0}^{n-1}n-\sum_{i=0}^{n-1}i=
        \frac{2\cdot n(n-1)}{2}-\frac{(n-1)\cdot(n-1)}{2}=\\
        &=\frac{n^2-1}{2}  operaciones 
    \end{split}
    \end{gather*}
        
    \item [-] La línea 8 es $\mathcal{O}(1)$
    \item [-] Las líneas 9 y 10 dependiendo del lenguaje de programación y de
    la estructura elegida para almacenar las operaciones y el conjunto, podrían
    ser $\mathcal{O}(1)$ en caso de ser listas, y $\mathcal{O}(n)$ cada una en
    caso de tratarse de vectores como en \C++{C++}. En la implementación
    aportada se emplea un \C++{vector} de la \C++{STL} de tamaño fijo,
    luego la línea 10 se computaría como $\mathcal{O}(1)$. Por simplicidad
    también consideraremos la línea 9 como $\mathcal{O}(1)$, dado que podría
    programarse una lista enlazada dotada del operador \C++{[]} necesario
    para implementar el algoritmo \ref{Numeros} en \C++{C++}.
    
    \item [-] Desde las líneas 12 a 17, se trata de operaciones $\mathcal{O}(1)$
    \item [-] De nuevo sobre las líneas 21, 22, 23 puede decirse lo mismo que sobre
    las 9,10. Aquí las consideraremos $\mathcal{O}(1)$
    \item [-] En la línea 18 se llama recursivamente al algoritmo. Por tanto:
    
    $$T(n)=\left\{\begin{array}{lr}
                   \frac{n^2-1}{2}\cdot T(n-1) & n>2\\
                   \ \\
                   1 & n=1
                  \end{array}\right.$$
                  
    y se tiene que:
    
    \begin{gather}
    \begin{split}
     \label{efic}
    &T(n)=\frac{n^2-1}{2}\cdot T(n-1)= \frac{n^2-1}{2}\cdot\frac{(n-1)^2-1}{2}
    \cdot T(n-2)=\\ \nonumber
    &=\ldots=T(n-j-1)\cdot\frac{1}{2^{j+1}}\prod_{i=0}^j[(n-i)^2-1]
    \end{split}
    \end{gather}
    
    Tomando $j=n-3$ en (\ref{efic}), se tiene $T(n)=
    \mathcal{O}\left(\frac{(n!)^2}{2^n}\right)$
   \end{itemize}

   % Normalización e impresión de operaciones
   \newpage
   
\subsection{Algoritmo de normalización de operaciones}
   Al calcular una solución del problema de las cifras, se van acumulando las operaciones
   por las que se pasan hasta llegar a \C++{meta}. Sin embargo, es posible que algunas de
   estas operaciones no se hayan utilizado en el cálculo de la solución, es decir, que sean
   inservibles, aunque se haya pasado por ellas. El siguiente algoritmo identifica dichas
   operaciones inútiles y las elimina de la secuencia generada por el algoritmo \ref{Cifras},
   para obtener una solución que contenga únicamente las operaciones necesarias.


   %Puesto que en la lista de operaciones aparecen todas las operaciones necesarias
   %para llegar a \C++{meta} o a una aproximación a la misma, pero también pueden
   %más operaciones que las estrictamente necesarias, se presenta a continuación,
   %otro algoritmo para normalizar dichas operaciones en función del resultado obtenido
   %por el algoritmo \ref{Cifras}. Nótese que este algoritmo está destinado a mejorar una
   %solución válida ya obtenida para hacer que se muestre con las mínimas operaciones posibles.\\
   
   \begin{algorithm}[H]
   \begin{algorithmic}[1]
   
   	\IF{hay más de una \C++{Cuenta} en la lista de operaciones}
		\STATE{Llama al siguiente algoritmo, pasándole \C++{primero}
		y \C++{segundo} de la última \C++{Cuenta} efectuada, y como
		posición de escritura la penúltima de \C++{operaciones} (podría ser \C++{-1})}
	\ENDIF
	\REQUIRE \ \\
	    \C++{unaCuenta}, última cuenta necesaria en la lista\\
	    \C++{pos\_escribir}, posición anterior a la última normalizada\\
	
	\STATE{La \C++{Cuenta} a consultar es la que ocupa \C++{pos\_escribir}}
   % Falta comentar el caso base de la inducción
	\WHILE {No se hallen \C++{primero} y \C++{segundo} de \C++{unaCuenta}
   como resultado de otra \C++{Cuenta}, y quede alguna por consultar}
		\STATE{
			\IF {el resultado de la \C++{Cuenta} consultada es \C++{primero} o \C++{segundo}}
				\STATE{Marcarlo como encontrado}
				\STATE{Intercambiar la \C++{Cuenta} que ocupa la posición \C++{pos\_escribir} en 
				\C++{operaciones} por la \C++{Cuenta} consultada}
				\STATE{Decrementa \C++{pos\_escribir} y llama al algoritmo para la última
				\C++{Cuenta} consultada, y \C++{pos\_escribir} }
				\STATE{El índice a consultar es ahora \C++{pos\_escribir}}
			\ELSE
				\STATE{Decrementa el índice de la posición a consultar}
			\ENDIF
		}
	\ENDWHILE
   \end{algorithmic}
   \ \\
   Una vez normalizadas las operaciones:
   \begin{algorithmic}[1]
    \STATE{Se itera \C++{operaciones} desde el principio
    hasta el final de la lista}
    \PRINT{\C++{Cuenta} actual}
   \end{algorithmic}

   \caption{ALGORITMO DE NORMALIZACIÓN DE OPERACIONES}
   \label{Numeros}
   \end{algorithm}
   % Falta comentar cómo las operaciones para la mejor operación se copian a cada paso
   
   \newpage
   \subsection{Optimización}
   El algoritmo mostrado puede ser optimizado haciendo que no estudie los casos en los que la pareja ordenada \C++{(a,b)} no cumpla unos
   determinados requisitos, y no compute el resultado en los casos en los que éste no aportara nada a la resolución del algoritmo. Las
   siguientes comprobaciones están diseñadas para aplicarse antes de la línea 8 del algoritmo principal, donde se calcula la operación. 
   En caso de que se cumpla una de estas condiciones, puede pasarse a la siguiente iteración del bucle.\\
   \begin{itemize}
    \item [-] El resultado \C++{a (op) b} es igual a \C++{a} o \C++{b}, por lo que no aporta nada a la resolución.
    \item [-] El resultado es igual a 0, \C++{a (op) b == 0}, pero las operaciones con \C++{0} no aportan nada a la resolución.
    \item [-] El resultado es negativo, como consecuencia del desbordamiento o de restas no válidas, no debe ser usado.
    %\item [-] Uno de los números es igual a \C++{0}, no aporta nada a la resolución.
    \item [-] Como optimización previa, los números están ordenados. Se previene la duplicación de parejas y de casos.
    \item [-] Como optimización previa, se descartan divisiones no enteras o restas negativas, que no pueden usarse en la resolución.
   \end{itemize}

   
   \newpage
   \subsection{Algoritmo alternativo}
   Se propone otro algoritmo destinado a dar una orientación alternativa a la solución del problema. El algoritmo principal se centra
   en buscar una solución siguiendo un sólo camino cada vez, y ahondando en la recursividad antes de volver a buscar por otro camino.
   Este algoritmo buscar avanzar de forma uniforme por todos los caminos posibles hacia la solución.
   
   Usamos varios grupos de \C++{numeros} y una cola de ellos, llamada \C++{colaGrupos}
   \begin{algorithm}[H]
   \begin{algorithmic}[1]
   
	\STATE{Insertamos el grupo de números iniciales en \C++{colaGrupos}}
	\WHILE{\C++{colaGrupos.noVacía()}}
	    \STATE{saca un grupo de la cola, \C++{numeros}}
	    \FOR{cada pareja ordenada \C++{(a,b)} en \C++{numeros}}
	    \FOR{cada operación \C++{op} en \C++{[+,*,-,/]}}
		\IF { \C++{a (op) b} es posible }
				\STATE { Computa \C++{a (op) b}}
				\IF { $|\C++{meta - a (op) b}| < |\C++{meta - mejor\_aprox}|$  }
				\STATE {$\C++{mejor\_aprox := meta}$}
				\IF {$\C++{mejor\_aprox == meta}$}
				\RETURN \TRUE
				\ENDIF
				\ENDIF
				
				\STATE { Crea un nuevo grupo por copia \C++{nuevos(numeros)}}
				\STATE { Retira \C++{a,b} de \C++{nuevos} }
				\STATE { Introduce \C++{a (op) b} en \C++ {nuevos} }
				\STATE { Introduce \C++{nuevos} en \C++{colaGrupos}}
				
		\ENDIF
	    \ENDFOR
	    \ENDFOR
	\ENDWHILE
   
   \end{algorithmic}
   \caption{ALGORITMO ALTERNATIVO}
   \label{Alternativo}
   \end{algorithm}
   
   Las optimizaciones serían paralelas a las del algoritmo principal. Se indica además, la posibilidad de mantener
   una cola con prioridad, facilitando el paso según tamaño del grupo o magnitud de sus números; y que además, elimine
   grupos repetidos aprovechando las comparaciones necesarias para mantener el orden.
   
   \newpage
   \section{Implementación}
   Mostramos ahora cómo podría implementarse el algoritmo propuesto, aplicando sobre él las optimizaciones sugeridas.
   Con esta implementación hemos querido comprobar el correcto funcionamiento y eficiencia del algoritmo. Para la mayoría
   de los casos, el tiempo requerido no excede los pocos segundos, estando el máximo en torno a los 20 segundos. \\
   
   % Habría que mostrar todo el código y dividirlo en los distintos ficheros.
   
   % Generator: GNU source-highlight, by Lorenzo Bettini, http://www.gnu.org/software/src-highlite
   \small
   \texttt{% Generator: GNU source-highlight, by Lorenzo Bettini, http://www.gnu.org/software/src-highlite
   \noindent
   \mbox{}\texttt{\textcolor{Black}{001:}} \textbf{\textcolor{RoyalBlue}{\#include}}\ \texttt{\textcolor{Red}{"{}cifras.h"{}}} \\
   \mbox{}\texttt{\textcolor{Black}{002:}} \textbf{\textcolor{Blue}{using}}\ \textbf{\textcolor{Blue}{namespace}}\ std\textcolor{BrickRed}{;} \\
   \mbox{}\texttt{\textcolor{Black}{003:}} \textbf{\textcolor{Blue}{typedef}}\ \textcolor{ForestGreen}{int}\ \textcolor{BrickRed}{(*}Operacion\textcolor{BrickRed}{)(}\textcolor{ForestGreen}{int}\ a\textcolor{BrickRed}{,}\ \textcolor{ForestGreen}{int}\ b\textcolor{BrickRed}{);}\  \\
   \mbox{}\texttt{\textcolor{Black}{004:}}  \\
   \mbox{}\texttt{\textcolor{Black}{005:}}  \\
   \mbox{}\texttt{\textcolor{Black}{006:}} Cifras\textcolor{BrickRed}{::}\textbf{\textcolor{Black}{Cifras}}\ \textcolor{BrickRed}{(}\textcolor{TealBlue}{vector\textless{}int\textgreater{}}\ introducidos\textcolor{BrickRed}{)}\ \textcolor{Red}{\{} \\
   \mbox{}\texttt{\textcolor{Black}{007:}} \textbf{\textcolor{RoyalBlue}{\ \ \ \ \#ifndef}}\ GRUPOS \\
   \mbox{}\texttt{\textcolor{Black}{008:}} \ \ \ \ \textit{\textcolor{Brown}{//\ Primera\ aproximación}} \\
   \mbox{}\texttt{\textcolor{Black}{009:}} \ \ \ \ mejor\ \textcolor{BrickRed}{=}\ \textcolor{BrickRed}{-}\textcolor{Purple}{1}\textcolor{BrickRed}{;} \\
   \mbox{}\texttt{\textcolor{Black}{010:}} \textbf{\textcolor{RoyalBlue}{\ \ \ \ \#endif}} \\
   \mbox{}\texttt{\textcolor{Black}{011:}} \ \  \\
   \mbox{}\texttt{\textcolor{Black}{012:}} \textbf{\textcolor{RoyalBlue}{\ \ \ \ \#ifdef}}\ GRUPOS \\
   \mbox{}\texttt{\textcolor{Black}{013:}} \ \ \ \ \textit{\textcolor{Brown}{//\ Números\ marcados}} \\
   \mbox{}\texttt{\textcolor{Black}{014:}} \ \ \ \ total$\_$encontrados\ \textcolor{BrickRed}{=}\ \textcolor{Purple}{0}\textcolor{BrickRed}{;} \\
   \mbox{}\texttt{\textcolor{Black}{015:}} \ \ \ \ \textcolor{TealBlue}{vector\textless{}bool\textgreater{}}\ \textbf{\textcolor{Black}{encontrado$\_$inicial}}\textcolor{BrickRed}{(}BUSCADOS\textcolor{BrickRed}{);} \\
   \mbox{}\texttt{\textcolor{Black}{016:}} \ \ \ \ \textbf{\textcolor{Blue}{for}}\ \textcolor{BrickRed}{(}\textcolor{ForestGreen}{int}\ i\textcolor{BrickRed}{=}\textcolor{Purple}{0}\textcolor{BrickRed}{;}\ i\textcolor{BrickRed}{\textless{}}BUSCADOS\textcolor{BrickRed}{;}\ \textcolor{BrickRed}{++}i\textcolor{BrickRed}{)} \\
   \mbox{}\texttt{\textcolor{Black}{017:}} \ \ \ \ \ \ \ \ encontrado$\_$inicial\textcolor{BrickRed}{[}i\textcolor{BrickRed}{]}\ \textcolor{BrickRed}{=}\ \textbf{\textcolor{Blue}{false}}\textcolor{BrickRed}{;} \\
   \mbox{}\texttt{\textcolor{Black}{018:}} \ \ \ \ encontrado\ \textcolor{BrickRed}{=}\ encontrado$\_$inicial\textcolor{BrickRed}{;} \\
   \mbox{}\texttt{\textcolor{Black}{019:}} \ \  \\
   \mbox{}\texttt{\textcolor{Black}{020:}} \ \ \ \ \textit{\textcolor{Brown}{//\ El\ cero\ se\ marca\ por\ defecto}} \\
   \mbox{}\texttt{\textcolor{Black}{021:}} \ \ \ \ \textbf{\textcolor{Black}{marcar}}\textcolor{BrickRed}{(}\textcolor{Purple}{0}\textcolor{BrickRed}{);} \\
   \mbox{}\texttt{\textcolor{Black}{022:}} \textbf{\textcolor{RoyalBlue}{\ \ \ \ \#endif}} \\
   \mbox{}\texttt{\textcolor{Black}{023:}} \ \  \\
   \mbox{}\texttt{\textcolor{Black}{024:}} \ \ \ \ \textit{\textcolor{Brown}{//\ Introduce\ los\ números\ en\ la\ doble\ cola.}} \\
   \mbox{}\texttt{\textcolor{Black}{025:}} \ \ \ \ \textcolor{TealBlue}{vector\textless{}int\textgreater{}}\ numeros\textcolor{BrickRed}{;} \\
   \mbox{}\texttt{\textcolor{Black}{026:}} \ \ \ \ \textcolor{ForestGreen}{int}\ size\ \textcolor{BrickRed}{=}\ introducidos\textcolor{BrickRed}{.}\textbf{\textcolor{Black}{size}}\textcolor{BrickRed}{();} \\
   \mbox{}\texttt{\textcolor{Black}{027:}} \ \ \ \ \textbf{\textcolor{Blue}{for}}\ \textcolor{BrickRed}{(}\textcolor{ForestGreen}{int}\ i\textcolor{BrickRed}{=}\textcolor{Purple}{0}\textcolor{BrickRed}{;}\ i\textcolor{BrickRed}{\textless{}}size\textcolor{BrickRed}{;}\ \textcolor{BrickRed}{++}i\textcolor{BrickRed}{)} \\
   \mbox{}\texttt{\textcolor{Black}{028:}} \ \ \ \ \ \ \ \ numeros\textcolor{BrickRed}{.}\textbf{\textcolor{Black}{push$\_$back}}\textcolor{BrickRed}{(}introducidos\textcolor{BrickRed}{[}i\textcolor{BrickRed}{]);} \\
   \mbox{}\texttt{\textcolor{Black}{029:}} \ \  \\
   \mbox{}\texttt{\textcolor{Black}{030:}} \ \ \ \ \textbf{\textcolor{Blue}{this}}\textcolor{BrickRed}{-\textgreater{}}numeros\ \textcolor{BrickRed}{=}\ numeros\textcolor{BrickRed}{;} \\
   \mbox{}\texttt{\textcolor{Black}{031:}} \textcolor{Red}{\}} \\
   \mbox{}\texttt{\textcolor{Black}{032:}}  \\
   \mbox{}\texttt{\textcolor{Black}{033:}} \textcolor{ForestGreen}{bool}\ Cifras\textcolor{BrickRed}{::}\textbf{\textcolor{Black}{resuelve}}\ \textcolor{BrickRed}{(}\textcolor{ForestGreen}{int}\ meta\textcolor{BrickRed}{)}\ \textcolor{Red}{\{} \\
   \mbox{}\texttt{\textcolor{Black}{034:}} \ \ \ \ \textit{\textcolor{Brown}{//\ Empieza\ comprobando\ que\ el\ número\ buscado\ no\ esté\ entre\ los\ dados.}} \\
   \mbox{}\texttt{\textcolor{Black}{035:}} \ \ \ \ \textbf{\textcolor{Blue}{for}}\ \textcolor{BrickRed}{(}vector\textcolor{BrickRed}{\textless{}}\textcolor{ForestGreen}{int}\textcolor{BrickRed}{\textgreater{}::}\textcolor{TealBlue}{iterator}\ it\textcolor{BrickRed}{=}numeros\textcolor{BrickRed}{.}\textbf{\textcolor{Black}{begin}}\textcolor{BrickRed}{();}\ it\ \textcolor{BrickRed}{!=}\ numeros\textcolor{BrickRed}{.}\textbf{\textcolor{Black}{end}}\textcolor{BrickRed}{();}\ \textcolor{BrickRed}{++}it\textcolor{BrickRed}{)}\textcolor{Red}{\{} \\
   \mbox{}\texttt{\textcolor{Black}{036:}} \textbf{\textcolor{RoyalBlue}{\ \ \ \ \ \ \ \ \#ifndef}}\ GRUPOS \\
   \mbox{}\texttt{\textcolor{Black}{037:}} \ \ \ \ \ \ \ \ \textbf{\textcolor{Blue}{if}}\ \textcolor{BrickRed}{(*}it\ \textcolor{BrickRed}{==}\ meta\textcolor{BrickRed}{)}\ \textcolor{Red}{\{} \\
   \mbox{}\texttt{\textcolor{Black}{038:}} \ \ \ \ \ \ \ \ \ \ \ \ \textcolor{TealBlue}{Cuenta}\ encontrada\ \textcolor{BrickRed}{=}\ \textcolor{Red}{\{}meta\textcolor{BrickRed}{,}\ meta\textcolor{BrickRed}{,}\ \texttt{\textcolor{Red}{'\ '}}\textcolor{Red}{\}}\textcolor{BrickRed}{;} \\
   \mbox{}\texttt{\textcolor{Black}{039:}} \ \ \ \ \ \ \ \ \ \ \ \ mejor$\_$operaciones\textcolor{BrickRed}{.}\textbf{\textcolor{Black}{push$\_$back}}\textcolor{BrickRed}{(}encontrada\textcolor{BrickRed}{);} \\
   \mbox{}\texttt{\textcolor{Black}{040:}} \ \ \ \ \ \ \ \ \ \ \ \  \\
   \mbox{}\texttt{\textcolor{Black}{041:}} \ \ \ \ \ \ \ \ \ \ \ \ \textbf{\textcolor{Blue}{return}}\ \textbf{\textcolor{Blue}{true}}\textcolor{BrickRed}{;} \\
   \mbox{}\texttt{\textcolor{Black}{042:}} \ \ \ \ \ \ \ \ \textcolor{Red}{\}} \\
   \mbox{}\texttt{\textcolor{Black}{043:}} \textbf{\textcolor{RoyalBlue}{\ \ \ \ \ \ \ \ \#endif}} \\
   \mbox{}\texttt{\textcolor{Black}{044:}} \ \ \ \  \\
   \mbox{}\texttt{\textcolor{Black}{045:}} \textbf{\textcolor{RoyalBlue}{\ \ \ \ \ \ \ \ \#ifdef}}\ GRUPOS \\
   \mbox{}\texttt{\textcolor{Black}{046:}} \ \ \ \ \ \ \ \ \textbf{\textcolor{Black}{marcar}}\textcolor{BrickRed}{(*}it\textcolor{BrickRed}{);} \\
   \mbox{}\texttt{\textcolor{Black}{047:}} \textbf{\textcolor{RoyalBlue}{\ \ \ \ \ \ \ \ \#endif}} \\
   \mbox{}\texttt{\textcolor{Black}{048:}} \ \ \ \ \textcolor{Red}{\}} \\
   \mbox{}\texttt{\textcolor{Black}{049:}} \ \  \\
   \mbox{}\texttt{\textcolor{Black}{050:}} \ \ \ \ \textit{\textcolor{Brown}{//\ Resuelve\ de\ forma\ recursiva\ todas\ las\ posibilidades.}} \\
   \mbox{}\texttt{\textcolor{Black}{051:}} \ \ \ \ \textcolor{ForestGreen}{bool}\ resuelto\textcolor{BrickRed}{=}\textbf{\textcolor{Black}{resuelve$\_$rec}}\textcolor{BrickRed}{(}meta\textcolor{BrickRed}{,}\ numeros\textcolor{BrickRed}{.}\textbf{\textcolor{Black}{size}}\textcolor{BrickRed}{());} \\
   \mbox{}\texttt{\textcolor{Black}{052:}} \ \  \\
   \mbox{}\texttt{\textcolor{Black}{053:}} \textbf{\textcolor{RoyalBlue}{\ \ \ \ \#ifndef}}\ GRUPOS \\
   \mbox{}\texttt{\textcolor{Black}{054:}} \ \ \ \ \textbf{\textcolor{Black}{normalizaOperaciones}}\textcolor{BrickRed}{();} \\
   \mbox{}\texttt{\textcolor{Black}{055:}} \textbf{\textcolor{RoyalBlue}{\ \ \ \ \#endif}} \\
   \mbox{}\texttt{\textcolor{Black}{056:}} \ \  \\
   \mbox{}\texttt{\textcolor{Black}{057:}} \ \ \ \ \textbf{\textcolor{Blue}{return}}\ resuelto\textcolor{BrickRed}{;} \\
   \mbox{}\texttt{\textcolor{Black}{058:}} \textcolor{Red}{\}} \\
   \mbox{}\texttt{\textcolor{Black}{059:}}  \\
   \mbox{}\texttt{\textcolor{Black}{060:}} \textcolor{ForestGreen}{bool}\ Cifras\textcolor{BrickRed}{::}\textbf{\textcolor{Black}{resuelve$\_$rec}}\ \textcolor{BrickRed}{(}\textcolor{ForestGreen}{int}\ meta\textcolor{BrickRed}{,}\ \textcolor{ForestGreen}{int}\ size\textcolor{BrickRed}{)}\ \textcolor{Red}{\{} \\
   \mbox{}\texttt{\textcolor{Black}{061:}} \ \ \ \ \textit{\textcolor{Brown}{//\ Operaciones}} \\
   \mbox{}\texttt{\textcolor{Black}{062:}} \ \ \ \ \textcolor{TealBlue}{Operacion}\ calcula\textcolor{BrickRed}{[]}\ \textcolor{BrickRed}{=}\ \textcolor{Red}{\{} \\
   \mbox{}\texttt{\textcolor{Black}{063:}} \ \ \ \ \ \ \ \ \textcolor{BrickRed}{[](}\textcolor{ForestGreen}{int}\ a\textcolor{BrickRed}{,}\ \textcolor{ForestGreen}{int}\ b\textcolor{BrickRed}{)}\textcolor{Red}{\{}\ \textbf{\textcolor{Blue}{return}}\ a\textcolor{BrickRed}{-}b\textcolor{BrickRed}{;}\ \textcolor{Red}{\}}\textcolor{BrickRed}{,} \\
   \mbox{}\texttt{\textcolor{Black}{064:}} \ \ \ \ \ \ \ \ \textcolor{BrickRed}{[](}\textcolor{ForestGreen}{int}\ a\textcolor{BrickRed}{,}\ \textcolor{ForestGreen}{int}\ b\textcolor{BrickRed}{)}\textcolor{Red}{\{}\ \textbf{\textcolor{Blue}{return}}\ a\textcolor{BrickRed}{/}b\textcolor{BrickRed}{;}\ \textcolor{Red}{\}}\textcolor{BrickRed}{,} \\
   \mbox{}\texttt{\textcolor{Black}{065:}} \ \ \ \ \ \ \ \ \textcolor{BrickRed}{[](}\textcolor{ForestGreen}{int}\ a\textcolor{BrickRed}{,}\ \textcolor{ForestGreen}{int}\ b\textcolor{BrickRed}{)}\textcolor{Red}{\{}\ \textbf{\textcolor{Blue}{return}}\ a\textcolor{BrickRed}{+}b\textcolor{BrickRed}{;}\ \textcolor{Red}{\}}\textcolor{BrickRed}{,} \\
   \mbox{}\texttt{\textcolor{Black}{066:}} \ \ \ \ \ \ \ \ \textcolor{BrickRed}{[](}\textcolor{ForestGreen}{int}\ a\textcolor{BrickRed}{,}\ \textcolor{ForestGreen}{int}\ b\textcolor{BrickRed}{)}\textcolor{Red}{\{}\ \textbf{\textcolor{Blue}{return}}\ a\textcolor{BrickRed}{*}b\textcolor{BrickRed}{;}\ \textcolor{Red}{\}} \\
   \mbox{}\texttt{\textcolor{Black}{067:}} \ \ \ \ \textcolor{Red}{\}}\textcolor{BrickRed}{;} \\
   \mbox{}\texttt{\textcolor{Black}{068:}} \ \ \ \  \\
   \mbox{}\texttt{\textcolor{Black}{069:}} \textbf{\textcolor{RoyalBlue}{\ \ \ \ \#ifndef}}\ GRUPOS \\
   \mbox{}\texttt{\textcolor{Black}{070:}} \ \ \ \ \textcolor{TealBlue}{Cuenta}\ opActual\textcolor{BrickRed}{;} \\
   \mbox{}\texttt{\textcolor{Black}{071:}} \textbf{\textcolor{RoyalBlue}{\ \ \ \ \#endif}} \\
   \mbox{}\texttt{\textcolor{Black}{072:}} \ \  \\
   \mbox{}\texttt{\textcolor{Black}{073:}} \ \ \ \ \textbf{\textcolor{Blue}{if}}\ \textcolor{BrickRed}{(}size\ \textcolor{BrickRed}{\textless{}}\ \textcolor{Purple}{2}\textcolor{BrickRed}{)}\ \textbf{\textcolor{Blue}{return}}\ \textbf{\textcolor{Blue}{false}}\textcolor{BrickRed}{;} \\
   \mbox{}\texttt{\textcolor{Black}{074:}} \ \  \\
   \mbox{}\texttt{\textcolor{Black}{075:}} \ \ \ \ \textit{\textcolor{Brown}{//\ Toma\ el\ primer\ número\ disponible}} \\
   \mbox{}\texttt{\textcolor{Black}{076:}} \ \ \ \ \textbf{\textcolor{Blue}{for}}\ \textcolor{BrickRed}{(}\textcolor{ForestGreen}{int}\ i\textcolor{BrickRed}{=}\textcolor{Purple}{0}\textcolor{BrickRed}{;}\ i\textcolor{BrickRed}{\textless{}}size\textcolor{BrickRed}{-}\textcolor{Purple}{1}\textcolor{BrickRed}{;}\ \textcolor{BrickRed}{++}i\textcolor{BrickRed}{)}\ \textcolor{Red}{\{} \\
   \mbox{}\texttt{\textcolor{Black}{077:}} \ \ \ \ \ \ \ \ \textcolor{ForestGreen}{int}\ a\ \textcolor{BrickRed}{=}\ numeros\textcolor{BrickRed}{[}i\textcolor{BrickRed}{];} \\
   \mbox{}\texttt{\textcolor{Black}{078:}} \ \ \ \  \\
   \mbox{}\texttt{\textcolor{Black}{079:}} \ \ \ \ \ \ \ \ \textbf{\textcolor{Blue}{if}}\ \textcolor{BrickRed}{(!(}a\textcolor{BrickRed}{==}\textcolor{Purple}{0}\textcolor{BrickRed}{))} \\
   \mbox{}\texttt{\textcolor{Black}{080:}} \ \ \ \ \ \ \ \ \ \ \ \ numeros\textcolor{BrickRed}{[}i\textcolor{BrickRed}{]=}numeros\textcolor{BrickRed}{[}size\textcolor{BrickRed}{-}\textcolor{Purple}{1}\textcolor{BrickRed}{];} \\
   \mbox{}\texttt{\textcolor{Black}{081:}} \ \ \ \ \ \ \ \ \textbf{\textcolor{Blue}{else}} \\
   \mbox{}\texttt{\textcolor{Black}{082:}} \ \ \ \ \ \ \ \ \ \ \ \ \textbf{\textcolor{Blue}{continue}}\textcolor{BrickRed}{;} \\
   \mbox{}\texttt{\textcolor{Black}{083:}} \ \ \ \  \\
   \mbox{}\texttt{\textcolor{Black}{084:}} \ \ \ \ \ \ \ \ \textit{\textcolor{Brown}{//\ Toma\ el\ segundo\ número\ disponible}} \\
   \mbox{}\texttt{\textcolor{Black}{085:}} \ \ \ \ \ \ \ \ \textbf{\textcolor{Blue}{for}}\ \textcolor{BrickRed}{(}\textcolor{ForestGreen}{int}\ j\textcolor{BrickRed}{=}i\textcolor{BrickRed}{;}\ j\textcolor{BrickRed}{\textless{}}size\textcolor{BrickRed}{-}\textcolor{Purple}{1}\textcolor{BrickRed}{;}\ \textcolor{BrickRed}{++}j\textcolor{BrickRed}{)}\ \textcolor{Red}{\{} \\
   \mbox{}\texttt{\textcolor{Black}{086:}} \ \ \ \ \ \ \ \ \ \ \ \ \textcolor{ForestGreen}{int}\ b\ \textcolor{BrickRed}{=}\ numeros\textcolor{BrickRed}{[}j\textcolor{BrickRed}{];} \\
   \mbox{}\texttt{\textcolor{Black}{087:}} \ \ \ \ \ \ \ \ \ \ \ \ \textbf{\textcolor{Blue}{if}}\ \textcolor{BrickRed}{(}b\ \textcolor{BrickRed}{!=}\ \textcolor{Purple}{0}\textcolor{BrickRed}{)} \\
   \mbox{}\texttt{\textcolor{Black}{088:}} \ \ \ \ \ \ \ \ \ \ \ \ \ \ \ \ numeros\textcolor{BrickRed}{[}j\textcolor{BrickRed}{]}\ \textcolor{BrickRed}{=}\ numeros\textcolor{BrickRed}{[}size\textcolor{BrickRed}{-}\textcolor{Purple}{2}\textcolor{BrickRed}{];} \\
   \mbox{}\texttt{\textcolor{Black}{089:}} \ \ \ \ \ \ \ \ \ \ \ \ \textbf{\textcolor{Blue}{else}} \\
   \mbox{}\texttt{\textcolor{Black}{090:}} \ \ \ \ \ \ \ \ \ \ \ \ \ \ \ \ \textbf{\textcolor{Blue}{continue}}\textcolor{BrickRed}{;} \\
   \mbox{}\texttt{\textcolor{Black}{091:}} \ \ \ \ \ \  \\
   \mbox{}\texttt{\textcolor{Black}{092:}} \ \ \ \ \ \ \ \ \ \ \ \ \textit{\textcolor{Brown}{//\ Y\ prueba\ sobre\ ellos\ todas\ las\ operaciones}} \\
   \mbox{}\texttt{\textcolor{Black}{093:}} \ \ \ \ \ \ \ \ \ \ \ \ \textbf{\textcolor{Blue}{for}}\ \textcolor{BrickRed}{(}\textcolor{ForestGreen}{int}\ op\textcolor{BrickRed}{=}\textcolor{Purple}{0}\textcolor{BrickRed}{;}\ op\textcolor{BrickRed}{\textless{}}NOP\textcolor{BrickRed}{;}\ \textcolor{BrickRed}{++}op\textcolor{BrickRed}{)}\ \textcolor{Red}{\{} \\
   \mbox{}\texttt{\textcolor{Black}{094:}} \ \ \ \ \ \ \ \ \ \ \ \ \ \ \ \ \textit{\textcolor{Brown}{//\ Cogemos\ siempre\ c\ como\ el\ mayor\ de\ ambos}} \\
   \mbox{}\texttt{\textcolor{Black}{095:}} \ \ \ \ \ \ \ \ \ \ \ \ \ \ \ \ \textcolor{ForestGreen}{int}\ c\textcolor{BrickRed}{=(}a\textcolor{BrickRed}{\textgreater{}}b\textcolor{BrickRed}{?}a\textcolor{BrickRed}{:}b\textcolor{BrickRed}{),}\ d\textcolor{BrickRed}{=(}c\textcolor{BrickRed}{==}a\textcolor{BrickRed}{?}b\textcolor{BrickRed}{:}a\textcolor{BrickRed}{);} \\
   \mbox{}\texttt{\textcolor{Black}{096:}} \ \ \ \ \ \ \ \  \\
   \mbox{}\texttt{\textcolor{Black}{097:}} \ \ \ \ \ \ \ \ \ \ \ \ \ \ \ \ \textit{\textcolor{Brown}{//\ Comprueba\ que\ la\ operación\ sea\ válida}} \\
   \mbox{}\texttt{\textcolor{Black}{098:}} \ \ \ \ \ \ \ \ \ \ \ \ \ \ \ \ \textcolor{ForestGreen}{bool}\ indivisible\ \textcolor{BrickRed}{=}\ \textcolor{BrickRed}{((}c\textcolor{BrickRed}{\%}d\ \textcolor{BrickRed}{!=}\ \textcolor{Purple}{0}\textcolor{BrickRed}{)}\ \textcolor{TealBlue}{and}\ op\textcolor{BrickRed}{==}DIV\textcolor{BrickRed}{);} \\
   \mbox{}\texttt{\textcolor{Black}{099:}} \ \ \ \ \ \ \ \ \ \ \ \ \ \ \ \ \textbf{\textcolor{Blue}{if}}\ \textcolor{BrickRed}{(}indivisible\textcolor{BrickRed}{)} \\
   \mbox{}\texttt{\textcolor{Black}{100:}} \ \ \ \ \ \ \ \ \ \ \ \ \ \ \ \ \ \ \ \ \textbf{\textcolor{Blue}{continue}}\textcolor{BrickRed}{;} \\
   \mbox{}\texttt{\textcolor{Black}{101:}} \ \ \ \ \ \ \ \  \\
   \mbox{}\texttt{\textcolor{Black}{102:}} \ \ \ \ \ \ \ \ \ \ \ \ \ \ \ \ \textit{\textcolor{Brown}{//\ Comprueba\ que\ la\ operación\ sea\ útil}} \\
   \mbox{}\texttt{\textcolor{Black}{103:}} \ \ \ \ \ \ \ \ \ \ \ \ \ \ \ \ \textcolor{ForestGreen}{int}\ resultado\ \textcolor{BrickRed}{=}\ calcula\textcolor{BrickRed}{[}op\textcolor{BrickRed}{](}c\textcolor{BrickRed}{,}d\textcolor{BrickRed}{);} \\
   \mbox{}\texttt{\textcolor{Black}{104:}} \ \ \ \ \ \ \ \ \ \ \ \ \ \ \ \ \textcolor{ForestGreen}{bool}\ trivial\ \textcolor{BrickRed}{=}\ \textcolor{BrickRed}{(}resultado\ \textcolor{BrickRed}{==}\ a\ \textcolor{TealBlue}{or}\ resultado\ \textcolor{BrickRed}{==}\ b\textcolor{BrickRed}{);} \\
   \mbox{}\texttt{\textcolor{Black}{105:}} \ \ \ \ \ \ \ \ \ \ \ \ \ \ \ \ \textcolor{ForestGreen}{bool}\ zero\ \textcolor{BrickRed}{=}\ \textcolor{BrickRed}{(}resultado\ \textcolor{BrickRed}{==}\ \textcolor{Purple}{0}\textcolor{BrickRed}{);} \\
   \mbox{}\texttt{\textcolor{Black}{106:}} \ \ \ \ \ \ \ \ \ \ \ \ \ \ \ \ \textcolor{ForestGreen}{bool}\ overflow\ \textcolor{BrickRed}{=}\ \textcolor{BrickRed}{(}resultado\ \textcolor{BrickRed}{\textless{}}\ \textcolor{Purple}{0}\textcolor{BrickRed}{);} \\
   \mbox{}\texttt{\textcolor{Black}{107:}} \ \ \ \ \ \ \ \ \ \ \ \ \ \ \ \ \textbf{\textcolor{Blue}{if}}\ \textcolor{BrickRed}{(}trivial\ or\ overflow\ \textcolor{TealBlue}{or}\ zero\textcolor{BrickRed}{)} \\
   \mbox{}\texttt{\textcolor{Black}{108:}} \ \ \ \ \ \ \ \ \ \ \ \ \ \ \ \ \ \ \ \ \textbf{\textcolor{Blue}{continue}}\textcolor{BrickRed}{;} \\
   \mbox{}\texttt{\textcolor{Black}{109:}} \ \ \ \ \ \ \ \ \ \  \\
   \mbox{}\texttt{\textcolor{Black}{110:}} \ \ \ \ \ \ \ \ \ \ \ \ \ \ \ \ \textit{\textcolor{Brown}{//\ Calcula\ y\ guarda\ la\ operación.}} \\
   \mbox{}\texttt{\textcolor{Black}{111:}} \textbf{\textcolor{RoyalBlue}{\ \ \ \ \ \ \ \ \ \ \ \ \ \ \ \ \#ifndef}}\ GRUPOS\ \ \ \ \ \ \ \ \ \  \\
   \mbox{}\texttt{\textcolor{Black}{112:}} \ \ \ \ \ \ \ \ \ \ \ \ \ \ \ \ opActual\ \textcolor{BrickRed}{=}\ \textcolor{Red}{\{}c\textcolor{BrickRed}{,}\ d\textcolor{BrickRed}{,}\ SIMBOLOS\textcolor{BrickRed}{[}op\textcolor{BrickRed}{],}\ resultado\textcolor{Red}{\}}\textcolor{BrickRed}{;} \\
   \mbox{}\texttt{\textcolor{Black}{113:}} \ \ \ \ \ \ \ \ \ \ \ \ \ \ \ \ operaciones\textcolor{BrickRed}{.}\textbf{\textcolor{Black}{push$\_$back}}\textcolor{BrickRed}{(}opActual\textcolor{BrickRed}{);} \\
   \mbox{}\texttt{\textcolor{Black}{114:}} \ \ \ \ \ \ \ \ \ \ \ \  \\
   \mbox{}\texttt{\textcolor{Black}{115:}} \ \ \ \ \ \ \ \ \ \ \ \ \ \ \ \ \textit{\textcolor{Brown}{//\ Intenta\ resolver\ o\ mejorar\ con\ el\ nuevo\ número,\ sin\ pasarse}} \\
   \mbox{}\texttt{\textcolor{Black}{116:}} \ \ \ \ \ \ \ \ \ \ \ \ \ \ \ \ \textbf{\textcolor{Blue}{if}}\ \textcolor{BrickRed}{(}\textbf{\textcolor{Black}{abs}}\textcolor{BrickRed}{(}meta\textcolor{BrickRed}{-}resultado\textcolor{BrickRed}{)}\ \textcolor{BrickRed}{\textless{}}\ \textbf{\textcolor{Black}{abs}}\textcolor{BrickRed}{(}meta\textcolor{BrickRed}{-}mejor\textcolor{BrickRed}{))}\ \textcolor{Red}{\{} \\
   \mbox{}\texttt{\textcolor{Black}{117:}} \ \ \ \ \ \ \ \ \ \ \ \ \ \ \ \ \ \ \ \ mejor\ \textcolor{BrickRed}{=}\ resultado\textcolor{BrickRed}{;} \\
   \mbox{}\texttt{\textcolor{Black}{118:}} \ \ \ \ \ \ \ \ \ \ \ \ \ \ \ \ \ \ \ \ mejor$\_$operaciones\ \textcolor{BrickRed}{=}\ operaciones\textcolor{BrickRed}{;} \\
   \mbox{}\texttt{\textcolor{Black}{119:}} \ \ \ \ \ \ \ \ \ \  \\
   \mbox{}\texttt{\textcolor{Black}{120:}} \ \ \ \ \ \ \ \ \ \ \ \ \ \ \ \ \ \ \ \ \textbf{\textcolor{Blue}{if}}\ \textcolor{BrickRed}{(}resultado\ \textcolor{BrickRed}{==}\ meta\textcolor{BrickRed}{)} \\
   \mbox{}\texttt{\textcolor{Black}{121:}} \ \ \ \ \ \ \ \ \ \ \ \ \ \ \ \ \ \ \ \ \ \ \ \ \textbf{\textcolor{Blue}{return}}\ \textbf{\textcolor{Blue}{true}}\textcolor{BrickRed}{;} \\
   \mbox{}\texttt{\textcolor{Black}{122:}} \ \ \ \ \ \ \ \ \ \ \ \ \ \ \ \ \textcolor{Red}{\}} \\
   \mbox{}\texttt{\textcolor{Black}{123:}} \textbf{\textcolor{RoyalBlue}{\ \ \ \ \ \ \ \ \ \ \ \ \ \ \ \ \#endif}} \\
   \mbox{}\texttt{\textcolor{Black}{124:}} \ \ \ \ \ \ \ \  \\
   \mbox{}\texttt{\textcolor{Black}{125:}} \ \ \ \ \ \ \ \ \ \ \ \ \ \ \ \ \textit{\textcolor{Brown}{//\ Marca\ el\ nuevo\ resultado\ y\ comprueba\ si\ están\ todos\ marcados}} \\
   \mbox{}\texttt{\textcolor{Black}{126:}} \textbf{\textcolor{RoyalBlue}{\ \ \ \ \ \ \ \ \ \ \ \ \ \ \ \ \#ifdef}}\ GRUPOS \\
   \mbox{}\texttt{\textcolor{Black}{127:}} \ \ \ \ \ \ \ \ \ \ \ \ \ \ \ \ \textbf{\textcolor{Blue}{if}}\ \textcolor{BrickRed}{(}\textbf{\textcolor{Black}{marcar}}\textcolor{BrickRed}{(}resultado\textcolor{BrickRed}{))} \\
   \mbox{}\texttt{\textcolor{Black}{128:}} \ \ \ \ \ \ \ \ \ \ \ \ \ \ \ \ \ \ \ \ \textbf{\textcolor{Blue}{return}}\ \textbf{\textcolor{Blue}{true}}\textcolor{BrickRed}{;} \\
   \mbox{}\texttt{\textcolor{Black}{129:}} \textbf{\textcolor{RoyalBlue}{\ \ \ \ \ \ \ \ \ \ \ \ \ \ \ \ \#endif}} \\
   \mbox{}\texttt{\textcolor{Black}{130:}} \ \ \ \ \ \ \ \ \ \ \ \ \ \  \\
   \mbox{}\texttt{\textcolor{Black}{131:}} \ \ \ \ \ \ \ \ \ \ \ \ \ \ \ \ \textit{\textcolor{Brown}{//\ Guarda\ el\ nuevo\ resultado\ y\ sigue\ buscando}} \\
   \mbox{}\texttt{\textcolor{Black}{132:}} \ \ \ \ \ \ \ \ \ \ \ \ \ \ \ \ numeros\textcolor{BrickRed}{[}size\textcolor{BrickRed}{-}\textcolor{Purple}{2}\textcolor{BrickRed}{]}\ \textcolor{BrickRed}{=}\ resultado\textcolor{BrickRed}{;} \\
   \mbox{}\texttt{\textcolor{Black}{133:}} \ \ \ \ \ \ \ \ \ \ \ \ \ \ \ \ \textbf{\textcolor{Blue}{if}}\ \textcolor{BrickRed}{(}\textbf{\textcolor{Black}{resuelve$\_$rec}}\textcolor{BrickRed}{(}meta\textcolor{BrickRed}{,}size\textcolor{BrickRed}{-}\textcolor{Purple}{1}\textcolor{BrickRed}{))} \\
   \mbox{}\texttt{\textcolor{Black}{134:}} \ \ \ \ \ \ \ \ \ \ \ \ \ \ \ \ \ \ \ \ \textbf{\textcolor{Blue}{return}}\ \textbf{\textcolor{Blue}{true}}\textcolor{BrickRed}{;} \\
   \mbox{}\texttt{\textcolor{Black}{135:}} \ \ \ \ \ \ \ \  \\
   \mbox{}\texttt{\textcolor{Black}{136:}} \textbf{\textcolor{RoyalBlue}{\ \ \ \ \ \ \ \ \ \ \ \ \ \ \ \ \#ifndef}}\ GRUPOS \\
   \mbox{}\texttt{\textcolor{Black}{137:}} \ \ \ \ \ \ \ \ \ \ \ \ \ \ \ \ \textit{\textcolor{Brown}{//Saca\ las\ operaciones}} \\
   \mbox{}\texttt{\textcolor{Black}{138:}} \ \ \ \ \ \ \ \ \ \ \ \ \ \ \ \ operaciones\textcolor{BrickRed}{.}\textbf{\textcolor{Black}{pop$\_$back}}\textcolor{BrickRed}{();} \\
   \mbox{}\texttt{\textcolor{Black}{139:}} \textbf{\textcolor{RoyalBlue}{\ \ \ \ \ \ \ \ \ \ \ \ \ \ \ \ \#endif}} \\
   \mbox{}\texttt{\textcolor{Black}{140:}} \ \ \ \ \ \ \ \ \ \ \ \ \textcolor{Red}{\}} \\
   \mbox{}\texttt{\textcolor{Black}{141:}} \ \ \ \ \ \ \ \ \ \ \ \ numeros\textcolor{BrickRed}{[}size\textcolor{BrickRed}{-}\textcolor{Purple}{2}\textcolor{BrickRed}{]=}numeros\textcolor{BrickRed}{[}j\textcolor{BrickRed}{];} \\
   \mbox{}\texttt{\textcolor{Black}{142:}} \ \ \ \ \ \ \ \ \ \ \ \ numeros\textcolor{BrickRed}{[}j\textcolor{BrickRed}{]=}b\textcolor{BrickRed}{;} \\
   \mbox{}\texttt{\textcolor{Black}{143:}} \ \ \ \ \ \ \ \ \textcolor{Red}{\}} \\
   \mbox{}\texttt{\textcolor{Black}{144:}} \ \ \ \ \ \ \ \ numeros\textcolor{BrickRed}{[}i\textcolor{BrickRed}{]=}a\textcolor{BrickRed}{;} \\
   \mbox{}\texttt{\textcolor{Black}{145:}} \ \ \ \ \textcolor{Red}{\}} \\
   \mbox{}\texttt{\textcolor{Black}{146:}} \ \ \ \ \textbf{\textcolor{Blue}{return}}\ \textbf{\textcolor{Blue}{false}}\textcolor{BrickRed}{;} \\
   \mbox{}\texttt{\textcolor{Black}{147:}} \textcolor{Red}{\}} \\
   \mbox{}\texttt{\textcolor{Black}{148:}}  \\
   \mbox{}\texttt{\textcolor{Black}{149:}} \textbf{\textcolor{RoyalBlue}{\#ifndef}}\ GRUPOS \\
   \mbox{}\texttt{\textcolor{Black}{150:}} \textcolor{ForestGreen}{void}\ Cifras\textcolor{BrickRed}{::}\textbf{\textcolor{Black}{escribeOperaciones}}\textcolor{BrickRed}{()}\ \textcolor{Red}{\{}\ \ \ \  \\
   \mbox{}\texttt{\textcolor{Black}{151:}} \ \ \ \ vector\textcolor{BrickRed}{\textless{}}Cuenta\textcolor{BrickRed}{\textgreater{}::}\textcolor{TealBlue}{iterator}\ it\textcolor{BrickRed}{;} \\
   \mbox{}\texttt{\textcolor{Black}{152:}} \ \  \\
   \mbox{}\texttt{\textcolor{Black}{153:}} \ \ \ \ \textbf{\textcolor{Blue}{for}}\textcolor{BrickRed}{(}it\textcolor{BrickRed}{=}mejor$\_$operaciones\textcolor{BrickRed}{.}\textbf{\textcolor{Black}{begin}}\textcolor{BrickRed}{();}\ it\textcolor{BrickRed}{!=}mejor$\_$operaciones\textcolor{BrickRed}{.}\textbf{\textcolor{Black}{end}}\textcolor{BrickRed}{();}\ it\textcolor{BrickRed}{++)}\textcolor{Red}{\{} \\
   \mbox{}\texttt{\textcolor{Black}{154:}} \ \ \ \ \ \ \ \ cout\ \textcolor{BrickRed}{\textless{}\textless{}}\ \textcolor{BrickRed}{*}it\ \textcolor{BrickRed}{\textless{}\textless{}}\ endl\textcolor{BrickRed}{;} \\
   \mbox{}\texttt{\textcolor{Black}{155:}} \ \ \ \ \textcolor{Red}{\}} \\
   \mbox{}\texttt{\textcolor{Black}{156:}} \textcolor{Red}{\}} \\
   \mbox{}\texttt{\textcolor{Black}{157:}} \textbf{\textcolor{RoyalBlue}{\#endif}} \\
   \mbox{}\texttt{\textcolor{Black}{158:}}  \\
   \mbox{}\texttt{\textcolor{Black}{159:}} \textbf{\textcolor{RoyalBlue}{\#ifdef}}\ GRUPOS \\
   \mbox{}\texttt{\textcolor{Black}{160:}} \textcolor{ForestGreen}{bool}\ Cifras\textcolor{BrickRed}{::}\textbf{\textcolor{Black}{marcar}}\textcolor{BrickRed}{(}\textcolor{ForestGreen}{int}\ n\textcolor{BrickRed}{)}\ \textcolor{Red}{\{} \\
   \mbox{}\texttt{\textcolor{Black}{161:}} \ \ \ \ \textbf{\textcolor{Blue}{if}}\ \textcolor{BrickRed}{(}n\textcolor{BrickRed}{\textless{}}\textcolor{Purple}{1000}\ and\ \textcolor{BrickRed}{!}encontrado\textcolor{BrickRed}{[}n\textcolor{BrickRed}{])}\ \textcolor{Red}{\{} \\
   \mbox{}\texttt{\textcolor{Black}{162:}} \ \ \ \ \ \ \ \ encontrado\textcolor{BrickRed}{[}n\textcolor{BrickRed}{]}\ \textcolor{BrickRed}{=}\ \textbf{\textcolor{Blue}{true}}\textcolor{BrickRed}{;} \\
   \mbox{}\texttt{\textcolor{Black}{163:}} \ \ \ \ \ \ \ \ \textcolor{BrickRed}{-\/-}total$\_$encontrados\textcolor{BrickRed}{;} \\
   \mbox{}\texttt{\textcolor{Black}{164:}} \ \ \ \ \ \ \ \ \textbf{\textcolor{Blue}{if}}\ \textcolor{BrickRed}{(}total$\_$encontrados\ \textcolor{BrickRed}{==}\ BUSCADOS\textcolor{BrickRed}{)} \\
   \mbox{}\texttt{\textcolor{Black}{165:}} \ \ \ \ \ \ \ \ \ \ \ \ \textbf{\textcolor{Blue}{return}}\ \textbf{\textcolor{Blue}{true}}\textcolor{BrickRed}{;} \\
   \mbox{}\texttt{\textcolor{Black}{166:}} \ \ \ \ \textcolor{Red}{\}} \\
   \mbox{}\texttt{\textcolor{Black}{167:}} \ \  \\
   \mbox{}\texttt{\textcolor{Black}{168:}} \ \ \ \ \textbf{\textcolor{Blue}{return}}\ \textbf{\textcolor{Blue}{false}}\textcolor{BrickRed}{;} \\
   \mbox{}\texttt{\textcolor{Black}{169:}} \textcolor{Red}{\}} \\
   \mbox{}\texttt{\textcolor{Black}{170:}} \textbf{\textcolor{RoyalBlue}{\#endif}} \\
   \mbox{}\texttt{\textcolor{Black}{171:}}  \\
   \mbox{}\texttt{\textcolor{Black}{172:}} \textbf{\textcolor{RoyalBlue}{\#ifdef}}\ GRUPOS \\
   \mbox{}\texttt{\textcolor{Black}{173:}} \textcolor{ForestGreen}{void}\ Cifras\textcolor{BrickRed}{::}\textbf{\textcolor{Black}{imprime$\_$restantes}}\ \textcolor{BrickRed}{()}\ \textcolor{Red}{\{} \\
   \mbox{}\texttt{\textcolor{Black}{174:}} \ \ \ \ \textbf{\textcolor{Blue}{for}}\ \textcolor{BrickRed}{(}\textcolor{ForestGreen}{int}\ i\textcolor{BrickRed}{=}\textcolor{Purple}{0}\textcolor{BrickRed}{;}\ i\textcolor{BrickRed}{\textless{}}BUSCADOS\textcolor{BrickRed}{;}\ \textcolor{BrickRed}{++}i\textcolor{BrickRed}{)} \\
   \mbox{}\texttt{\textcolor{Black}{175:}} \ \ \ \ \ \ \ \ \textbf{\textcolor{Blue}{if}}\ \textcolor{BrickRed}{(!}encontrado\textcolor{BrickRed}{[}i\textcolor{BrickRed}{])} \\
   \mbox{}\texttt{\textcolor{Black}{176:}} \ \ \ \ \ \ \ \ \ \ \ \ cout\ \textcolor{BrickRed}{\textless{}\textless{}}\ i\ \textcolor{BrickRed}{\textless{}\textless{}}\ \texttt{\textcolor{Red}{'\ '}}\textcolor{BrickRed}{;} \\
   \mbox{}\texttt{\textcolor{Black}{177:}} \ \ \ \ cout\ \textcolor{BrickRed}{\textless{}\textless{}}\ endl\textcolor{BrickRed}{;} \\
   \mbox{}\texttt{\textcolor{Black}{178:}} \textcolor{Red}{\}} \\
   \mbox{}\texttt{\textcolor{Black}{179:}} \textbf{\textcolor{RoyalBlue}{\#endif}} \\
   \mbox{}\texttt{\textcolor{Black}{180:}}  \\
   \mbox{}\texttt{\textcolor{Black}{181:}} \textbf{\textcolor{RoyalBlue}{\#ifdef}}\ GRUPOS \\
   \mbox{}\texttt{\textcolor{Black}{182:}} \textcolor{ForestGreen}{bool}\ Cifras\textcolor{BrickRed}{::}\textbf{\textcolor{Black}{todos$\_$marcados}}\ \textcolor{BrickRed}{()}\ \textcolor{Red}{\{} \\
   \mbox{}\texttt{\textcolor{Black}{183:}} \ \ \ \ \textcolor{ForestGreen}{bool}\ todos\ \textcolor{BrickRed}{=}\ \textbf{\textcolor{Blue}{true}}\textcolor{BrickRed}{;} \\
   \mbox{}\texttt{\textcolor{Black}{184:}} \ \ \ \ \textbf{\textcolor{Blue}{for}}\ \textcolor{BrickRed}{(}\textcolor{ForestGreen}{int}\ i\textcolor{BrickRed}{=}\textcolor{Purple}{0}\textcolor{BrickRed}{;}\ i\textcolor{BrickRed}{\textless{}}BUSCADOS\ \textcolor{TealBlue}{and}\ todos\textcolor{BrickRed}{;}\ \textcolor{BrickRed}{++}i\textcolor{BrickRed}{)} \\
   \mbox{}\texttt{\textcolor{Black}{185:}} \ \ \ \ \ \ \ \ todos\ \textcolor{BrickRed}{=}\ encontrado\textcolor{BrickRed}{[}i\textcolor{BrickRed}{];} \\
   \mbox{}\texttt{\textcolor{Black}{186:}} \ \ \ \ \textbf{\textcolor{Blue}{return}}\ todos\textcolor{BrickRed}{;} \\
   \mbox{}\texttt{\textcolor{Black}{187:}} \textcolor{Red}{\}} \\
   \mbox{}\texttt{\textcolor{Black}{188:}} \textbf{\textcolor{RoyalBlue}{\#endif}} \\
   \mbox{}\texttt{\textcolor{Black}{189:}}  \\
   \mbox{}\texttt{\textcolor{Black}{190:}} \textbf{\textcolor{RoyalBlue}{\#ifndef}}\ GRUPOS \\
   \mbox{}\texttt{\textcolor{Black}{191:}} \textcolor{ForestGreen}{void}\ Cifras\textcolor{BrickRed}{::}\textbf{\textcolor{Black}{normalizaOperaciones}}\textcolor{BrickRed}{()}\ \textcolor{Red}{\{} \\
   \mbox{}\texttt{\textcolor{Black}{192:}} \ \ \ \ \textcolor{ForestGreen}{int}\ size\ \textcolor{BrickRed}{=}\ mejor$\_$operaciones\textcolor{BrickRed}{.}\textbf{\textcolor{Black}{size}}\textcolor{BrickRed}{();} \\
   \mbox{}\texttt{\textcolor{Black}{193:}} \ \ \ \ \textcolor{ForestGreen}{int}\ pos$\_$escribir\ \textcolor{BrickRed}{=}\ size\ \textcolor{BrickRed}{-}\ \textcolor{Purple}{2}\textcolor{BrickRed}{;} \\
   \mbox{}\texttt{\textcolor{Black}{194:}} \ \ \ \  \\
   \mbox{}\texttt{\textcolor{Black}{195:}} \ \ \ \ \textbf{\textcolor{Blue}{if}}\ \textcolor{BrickRed}{(}pos$\_$escribir\ \textcolor{BrickRed}{\textgreater{}=}\ \textcolor{Purple}{0}\textcolor{BrickRed}{)}\textcolor{Red}{\{} \\
   \mbox{}\texttt{\textcolor{Black}{196:}} \ \ \ \ \ \ \ \ \textbf{\textcolor{Black}{buscaOperandos}}\ \textcolor{BrickRed}{(}mejor$\_$operaciones\textcolor{BrickRed}{[}size\textcolor{BrickRed}{-}\textcolor{Purple}{1}\textcolor{BrickRed}{],}pos$\_$escribir\textcolor{BrickRed}{);} \\
   \mbox{}\texttt{\textcolor{Black}{197:}} \ \  \\
   \mbox{}\texttt{\textcolor{Black}{198:}} \ \ \ \ \ \ \ \ mejor$\_$operaciones\textcolor{BrickRed}{.}\textbf{\textcolor{Black}{erase}}\textcolor{BrickRed}{(}mejor$\_$operaciones\textcolor{BrickRed}{.}\textbf{\textcolor{Black}{begin}}\textcolor{BrickRed}{(),} \\
   \mbox{}\texttt{\textcolor{Black}{199:}} \ \ \ \ \ \ \ \ \ \ \ \  mejor$\_$operaciones\textcolor{BrickRed}{.}\textbf{\textcolor{Black}{begin}}\textcolor{BrickRed}{()}\ \textcolor{BrickRed}{+}\ pos$\_$escribir\ \textcolor{BrickRed}{+}\ \textcolor{Purple}{1}\textcolor{BrickRed}{);}\ \ \  \\
   \mbox{}\texttt{\textcolor{Black}{200:}} \ \ \ \ \textcolor{Red}{\}} \\
   \mbox{}\texttt{\textcolor{Black}{201:}} \textcolor{Red}{\}} \\
   \mbox{}\texttt{\textcolor{Black}{202:}}  \\
   \mbox{}\texttt{\textcolor{Black}{203:}} \textcolor{ForestGreen}{void}\ Cifras\textcolor{BrickRed}{::}\textbf{\textcolor{Black}{buscaOperandos}}\textcolor{BrickRed}{(}\textcolor{TealBlue}{Cuenta}\ unaCuenta\textcolor{BrickRed}{,}\ \textcolor{ForestGreen}{int}\textcolor{BrickRed}{\&}\ pos$\_$escribir\textcolor{BrickRed}{)}\textcolor{Red}{\{} \\
   \mbox{}\texttt{\textcolor{Black}{204:}} \ \ \ \ \textcolor{ForestGreen}{bool}\ uno$\_$encontrado\textcolor{BrickRed}{=}\textbf{\textcolor{Blue}{false}}\textcolor{BrickRed}{,}\ otro$\_$encontrado\textcolor{BrickRed}{=}\textbf{\textcolor{Blue}{false}}\textcolor{BrickRed}{;} \\
   \mbox{}\texttt{\textcolor{Black}{205:}} \ \ \ \ \textcolor{ForestGreen}{int}\ j\ \textcolor{BrickRed}{=}\ pos$\_$escribir\textcolor{BrickRed}{;} \\
   \mbox{}\texttt{\textcolor{Black}{206:}} \ \ \ \ \textcolor{ForestGreen}{int}\ un$\_$operando\ \textcolor{BrickRed}{=}\ unaCuenta\textcolor{BrickRed}{.}primero\textcolor{BrickRed}{,} \\
   \mbox{}\texttt{\textcolor{Black}{207:}} \ \ \ \ \ \ \ \ otro$\_$operando\ \textcolor{BrickRed}{=}\ unaCuenta\textcolor{BrickRed}{.}segundo\textcolor{BrickRed}{;} \\
   \mbox{}\texttt{\textcolor{Black}{208:}} \ \ \ \ \ \ \ \  \\
   \mbox{}\texttt{\textcolor{Black}{209:}} \ \ \ \ \textbf{\textcolor{Blue}{while}}\ \textcolor{BrickRed}{((!}uno$\_$encontrado\ \textcolor{BrickRed}{$|$$|$}\ \textcolor{BrickRed}{!}otro$\_$encontrado\textcolor{BrickRed}{)}\ \textcolor{BrickRed}{\&\&}\ j\textcolor{BrickRed}{\textgreater{}=}\textcolor{Purple}{0}\textcolor{BrickRed}{)}\textcolor{Red}{\{} \\
   \mbox{}\texttt{\textcolor{Black}{210:}} \ \ \ \ \ \ \ \ \textbf{\textcolor{Blue}{if}}\ \textcolor{BrickRed}{((}mejor$\_$operaciones\textcolor{BrickRed}{[}j\textcolor{BrickRed}{].}resultado\ \textcolor{BrickRed}{==}\ un$\_$operando\textcolor{BrickRed}{)}\ \textcolor{BrickRed}{$|$$|$}\  \\
   \mbox{}\texttt{\textcolor{Black}{211:}} \ \ \ \ \ \ \ \ \ \ \ \ \textcolor{BrickRed}{(}mejor$\_$operaciones\textcolor{BrickRed}{[}j\textcolor{BrickRed}{].}resultado\ \textcolor{BrickRed}{==}\ otro$\_$operando\textcolor{BrickRed}{))}\textcolor{Red}{\{} \\
   \mbox{}\texttt{\textcolor{Black}{212:}} \ \ \ \ \ \  \\
   \mbox{}\texttt{\textcolor{Black}{213:}} \ \ \ \ \ \ \ \ \ \ \ \ \textbf{\textcolor{Blue}{if}}\ \textcolor{BrickRed}{(}uno$\_$encontrado\textcolor{BrickRed}{)} \\
   \mbox{}\texttt{\textcolor{Black}{214:}} \ \ \ \ \ \ \ \ \ \ \ \ \ \ \ \ otro$\_$encontrado\textcolor{BrickRed}{=}\textbf{\textcolor{Blue}{true}}\textcolor{BrickRed}{;} \\
   \mbox{}\texttt{\textcolor{Black}{215:}} \ \ \ \ \ \ \ \ \ \ \ \ \textbf{\textcolor{Blue}{else}} \\
   \mbox{}\texttt{\textcolor{Black}{216:}} \ \ \ \ \ \ \ \ \ \ \ \ \ \ \ \ uno$\_$encontrado\textcolor{BrickRed}{=}\textbf{\textcolor{Blue}{true}}\textcolor{BrickRed}{;} \\
   \mbox{}\texttt{\textcolor{Black}{217:}}  \\
   \mbox{}\texttt{\textcolor{Black}{218:}} \ \ \ \ \ \ \ \ \ \ \ \ \textcolor{TealBlue}{Cuenta}\ \textbf{\textcolor{Black}{aux}}\textcolor{BrickRed}{(}mejor$\_$operaciones\textcolor{BrickRed}{[}j\textcolor{BrickRed}{]);} \\
   \mbox{}\texttt{\textcolor{Black}{219:}} \ \ \ \ \ \ \ \ \ \ \ \ mejor$\_$operaciones\textcolor{BrickRed}{[}j\textcolor{BrickRed}{]=}mejor$\_$operaciones\textcolor{BrickRed}{[}pos$\_$escribir\textcolor{BrickRed}{];} \\
   \mbox{}\texttt{\textcolor{Black}{220:}} \ \ \ \ \ \ \ \ \ \ \ \ mejor$\_$operaciones\textcolor{BrickRed}{[}pos$\_$escribir\textcolor{BrickRed}{]=}aux\textcolor{BrickRed}{;} \\
   \mbox{}\texttt{\textcolor{Black}{221:}} \ \ \ \ \ \ \ \ \ \ \ \ pos$\_$escribir\textcolor{BrickRed}{-\/-;} \\
   \mbox{}\texttt{\textcolor{Black}{222:}} \ \ \ \ \ \  \\
   \mbox{}\texttt{\textcolor{Black}{223:}} \ \ \ \ \ \ \ \ \ \ \ \ \textbf{\textcolor{Black}{buscaOperandos}}\textcolor{BrickRed}{(}mejor$\_$operaciones\textcolor{BrickRed}{[}pos$\_$escribir\textcolor{BrickRed}{+}\textcolor{Purple}{1}\textcolor{BrickRed}{],}\ pos$\_$escribir\textcolor{BrickRed}{);} \\
   \mbox{}\texttt{\textcolor{Black}{224:}} \ \ \ \ \ \ \ \ \ \ \ \ j\ \textcolor{BrickRed}{=}\ pos$\_$escribir\textcolor{BrickRed}{;} \\
   \mbox{}\texttt{\textcolor{Black}{225:}} \ \ \ \ \ \ \ \ \textcolor{Red}{\}} \\
   \mbox{}\texttt{\textcolor{Black}{226:}} \ \ \ \ \ \ \ \ \textbf{\textcolor{Blue}{else}} \\
   \mbox{}\texttt{\textcolor{Black}{227:}} \ \ \ \ \ \ \ \ \ \ \ \ j\textcolor{BrickRed}{-\/-;} \\
   \mbox{}\texttt{\textcolor{Black}{228:}} \ \ \ \ \textcolor{Red}{\}} \\
   \mbox{}\texttt{\textcolor{Black}{229:}} \textcolor{Red}{\}} \\
   \mbox{}\texttt{\textcolor{Black}{230:}}  \\
   \mbox{}\texttt{\textcolor{Black}{231:}}  \\
   \mbox{}\texttt{\textcolor{Black}{232:}} std\textcolor{BrickRed}{::}ostream\textcolor{BrickRed}{\&}\ \textbf{\textcolor{Blue}{operator}}\textcolor{BrickRed}{\textless{}\textless{}(}std\textcolor{BrickRed}{::}ostream\textcolor{BrickRed}{\&}\ salida\textcolor{BrickRed}{,}\ \textbf{\textcolor{Blue}{const}}\ Cifras\textcolor{BrickRed}{::}Cuenta\textcolor{BrickRed}{\&}\ operacion\textcolor{BrickRed}{)}\textcolor{Red}{\{} \\
   \mbox{}\texttt{\textcolor{Black}{233:}} \ \ \ \ \textbf{\textcolor{Blue}{if}}\ \textcolor{BrickRed}{(}operacion\textcolor{BrickRed}{.}operador\ \textcolor{BrickRed}{!=}\ \texttt{\textcolor{Red}{'\ '}}\textcolor{BrickRed}{)} \\
   \mbox{}\texttt{\textcolor{Black}{234:}} \ \ \ \ \ \ \ \ salida\ \textcolor{BrickRed}{\textless{}\textless{}}\ operacion\textcolor{BrickRed}{.}primero\ \textcolor{BrickRed}{\textless{}\textless{}}\ operacion\textcolor{BrickRed}{.}operador\ \textcolor{BrickRed}{\textless{}\textless{}} \\
   \mbox{}\texttt{\textcolor{Black}{235:}} \ \ \ \ \ \ \ \ \ \ \ \ operacion\textcolor{BrickRed}{.}segundo\ \textcolor{BrickRed}{\textless{}\textless{}}\ \texttt{\textcolor{Red}{'='}}\ \textcolor{BrickRed}{\textless{}\textless{}}\ operacion\textcolor{BrickRed}{.}resultado\textcolor{BrickRed}{;} \\
   \mbox{}\texttt{\textcolor{Black}{236:}} \ \ \ \ \textbf{\textcolor{Blue}{else}}\  \\
   \mbox{}\texttt{\textcolor{Black}{237:}} \ \ \ \ \ \ \ \ salida\ \textcolor{BrickRed}{\textless{}\textless{}}\ operacion\textcolor{BrickRed}{.}primero\textcolor{BrickRed}{;} \\
   \mbox{}\texttt{\textcolor{Black}{238:}} \ \ \ \  \\
   \mbox{}\texttt{\textcolor{Black}{239:}} \ \ \ \ \textbf{\textcolor{Blue}{return}}\ salida\textcolor{BrickRed}{;} \\
   \mbox{}\texttt{\textcolor{Black}{240:}} \textcolor{Red}{\}} \\
   \mbox{}\texttt{\textcolor{Black}{241:}} \textbf{\textcolor{RoyalBlue}{\#endif}} \\
   \mbox{}\texttt{\textcolor{Black}{242:}} 
   }
   
   
   
   
   
   
\end{document}