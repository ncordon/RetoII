\documentclass[a4paper,10pt]{scrartcl}
\usepackage[utf8]{inputenc}
\usepackage{verbatim}
\usepackage{algorithm}
\usepackage{algorithmic}
\everymath{\displaystyle}
\floatname{algorithm}{Algoritmo}
\renewcommand{\listalgorithmname}{Lista de algoritmos}
\renewcommand{\algorithmicrequire}{\textbf{Entrada:}}
\renewcommand{\algorithmicensure}{\textbf{Salida:}}
\renewcommand{\algorithmicend}{\textbf{Fin}}
\renewcommand{\algorithmicif}{\textbf{Si}}
\renewcommand{\algorithmicthen}{\textbf{entonces}}
\renewcommand{\algorithmicelse}{\textbf{Si no}}
\renewcommand{\algorithmicelsif}{\algorithmicelse,\ \algorithmicif}
\renewcommand{\algorithmicendif}{\algorithmicend\ \algorithmicif}
\renewcommand{\algorithmicfor}{\textbf{Para }}
\renewcommand{\algorithmicforall}{\textbf{para todo}}
\renewcommand{\algorithmicdo}{\textbf{}}
\renewcommand{\algorithmicendfor}{\algorithmicend\ \algorithmicfor}
\renewcommand{\algorithmicwhile}{\textbf{mientras}}
\renewcommand{\algorithmicendwhile}{\algorithmicend\ \algorithmicwhile}
\renewcommand{\algorithmicloop}{\textbf{repetir}}
\renewcommand{\algorithmicendloop}{\algorithmicend\ \algorithmicloop}
\renewcommand{\algorithmicrepeat}{\textbf{repetir}}
\renewcommand{\algorithmicuntil}{\textbf{hasta que}}
\renewcommand{\algorithmicprint}{\textbf{imprimir}} 
\renewcommand{\algorithmicreturn}{\textbf{devolver}} 
\renewcommand{\algorithmictrue}{\textbf{true }} 
\renewcommand{\algorithmicfalse}{\textbf{false }} 
  % Variables de entorno en español
\setlength{\parindent}{0cm}

\newcommand{\objeto}[4]{\indent\underline{\textbf{TDA #1}}
            \begin{center}\begin{tabular}{|p{7cm}|}
                 \hline
                 \multicolumn{1}{|c|}{\texttt{#1}}\\
                 \hline
                  \\#2\\\\
                 \hline
                 \\#3\\\\
                 \hline
            \end{tabular}\end{center}
            \begin{itemize}
            #4
            \end{itemize}
               }
\def\C++#1{\texttt{#1}}
\def\T_{\texttt{T}}
%opening
\title{Reto II: Problema de las cifras}
\author{Francisco David Charte Luque \and 
        Ignacio Cordón Castillo \and
        Mario Román García}
\date{}
\begin{document}
\maketitle

Se emplea la siguiente notación genérica para la representación 
de un TDA abstracto:\\

\objeto{MiTDA}{
		\C++{- datos privados}\\
		\C++{- métodos privados}
	}{
		\C++{+ datos accesibles a través de la interfaz}\\
		\C++{+ métodos invocables desde la interfaz}
	}{\item Descripciones sobre el TDA}

\bigskip
Los TDA empleados en la resolución del problema de las cifras han sido:\\

\objeto{Cuenta}{}{
        \C++{+ primero}\\
        \C++{+ segundo}\\
        \C++{+ operador}\\
        \C++{+ resultado}
    }{
        \item \C++{primero} Número entero, representando el primer operando.
        \item \C++{segundo} Número entero, que representa el segundo operando.
        \item \C++{operador} Carácter que corresponde a la operación realizada sobre los números.
        \item \C++{resultado} Número entero, resultado de realizar la operación sobre \C++{primero} y \C++{segundo}.
    }
 %\newpage 
 \objeto{ProblemaCifras}{\C++{- numerosIniciales}\\
                        \C++{- numeros}\\
                        \C++{- operaciones}\\
                        \C++{- meta}\\
                        \C++{- operacionesPosibles}\\
                        }
                        {\C++{+ opera()}\\
                        \C++{+ resuelve()}}{
  \item \C++{numerosIniciales} Conjunto que almacena los enteros
   a partir de los que se pretende obtener \C++{meta}
  \item \C++{numeros} Lista sobre la que se realizarán todas las
   operaciones necesarias hasta llegar a una aproximación (o al número
   buscado exactamente) de \C++{meta}
  \item \C++{operaciones} Lista de objetos \C++{Cuenta} en los que
   se almacenarán las operaciones realizadas hasta llegar a \C++{meta},
   o a una aproximación a \C++{meta}
  \item \C++{meta} Entero positivo de 3 cifras a aproximar, y en
   caso de ser posible, hallar de forma exacta mediante operaciones sobre
   las cifras dadas iniciales
  \item \C++{operacionesPosibles} Conjunto que contiene todas las
   operaciones posibles aplicables \{\C++{+,*,-,/}\}
  \item \C++{opera()} Función que devuelve para dos operandos dados, el resultado
   de una operación determinada de entre \C++{operacionesPosibles} para ellos
  \item \C++{resuelve()} Función recursiva que selecciona parejas de cifras
   de \C++{numeros}, que introduce operados (con \C++{opera()}) en dicha lista, 
   para llamarse a sí misma e intentar llegar a \C++{meta}. Caso de no producir
   acierto, saca los números introducidos y devuelve los extraídos, y reitera con 
   otra pareja
  }
   
   \newpage
   % Recreo el algoritmo intentando usar el más alto nivel posible.
   
   El algoritmo propuesto para resolver el problema de las cifras es:\\
   
   \begin{algorithm}[H]
   \begin{algorithmic}[1]
   \REQUIRE \ \\
	    \C++{meta}, número a aproximar\\
	    \C++{numeros}, enteros aleatorios iniciales del conjunto\\
            \C++{size}, número de posiciones de la lista \C++{números}    
            
   \ENSURE \ \\
	   \TRUE {si logramos alcanzar exactamente \C++{meta} o es una
			 de las cifras de \C++{numeros}}\\
	   \FALSE {si sólo logramos una aproximación \C++{aprox} a \C++{meta}}
   
	\STATE { Inicializa \C++{mejor\_aprox} a -1. } 
	\IF{ Hay al menos dos cifras que seleccionar}
		\STATE{
		\FOR{cada pareja ordenada \C++{(a,b)} en \C++{numeros}}
			\STATE{ 
			\FOR{cada operación \C++{op} en \C++{[+,*,-,/]}} 
			\IF { \C++{a (op) b} es posible }
				\STATE { Computa la \C++{cuenta} }
				\STATE { Almacena la cuenta en la pila de cuentas }
				\STATE { Retira \C++{a,b} del conjunto de números }
				\STATE { Introduce \C++{a (op) b} en el conjunto de números }
				\IF { $\C++{0}\leq\C++{meta - a (op) b} \leq \C++{meta - mejor\_aprox}$  }
				\STATE {$\C++{mejor\_aprox := meta}$}
				\IF {$\C++{mejor\_aprox == meta}$}
				\RETURN \TRUE
				\ENDIF
				\ENDIF
				\IF { llamamos recursivamente al algoritmo sobre \C++{números} y devuelve \TRUE }
				\RETURN \TRUE
				\ENDIF
				\STATE { Retira la \C++{cuenta} de la pila de cuentas }
				\STATE { Retira \C++{a (op) b} del conjunto de números }
				\STATE { Reintroduce \C++{a,b} en el conjunto de números }
			\ENDIF
			\ENDFOR
			}
		\ENDFOR
		}
    \ELSE
		\STATE{Devolver \FALSE}
	\ENDIF
   \end{algorithmic}
   \caption{ALGORITMO DE CÁLCULO DEL NÚMERO DE 3 CIFRAS}
   \label{Cifras}
   \end{algorithm}
   
   Llamando $T(n)$ a la función que da la eficiencia del algoritmo \ref{Cifras},
   en función de la longitud del vector \C++{numeros}, esto es, de las cifras
   dadas para llegar a meta, se tiene:
   \begin{itemize} 
    \item [-] Desde las líneas 1 a 3, las operaciones realizadas son $\mathcal{O}(1)$
    \item [-] La selección de parejas \C++{(a,b)} de la línea 4 se hace mediante
    combinaciones ${a\choose b}$. Explícitamente, podemos observar como en
    la implementación en \C++{C++} adjunta, esto supone:
    
    $$\begin{array}{llr}
    \sum_{i=0}^{n-1}(n-i)=&\sum_{i=0}^{n-1}n-\sum_{i=0}^{n-1}i=&
    \frac{2\cdot n(n-1)}{2}-\frac{(n-1)\cdot(n-1)}{2}=\\
    =\frac{n^2-1}{2} & operaciones &
    \end{array}$$
    
    \item [-] La línea 8 es $\mathcal{O}(1)$
    \item [-] Las líneas 9 y 10 dependiendo del lenguaje de programación y de
    la estructura elegida para almacenar las operaciones y el conjunto, podrían
    ser $\mathcal{O}(1)$ en caso de ser listas, y $\mathcal{O}(n)$ cada una en
    caso de tratarse de vectores como en \C++{C++}. En la implementación
    aportada se emplea un \C++{vector} de la \C++{STL} de tamaño fijo,
    luego la línea 10 se computaría como $\mathcal{O}(1)$. Por simplicidad
    también consideraremos la línea 9 como $\mathcal{O}(1)$, dado que podría
    programarse una lista enlazada dotada del operador \C++{[]} necesario
    para implementar el algoritmo \ref{Numeros} en \C++{C++}.
    
    \item [-] Desde las líneas 12 a 17, se trata de operaciones $\mathcal{O}(1)$
    \item [-] De nuevo sobre las líneas 21, 22, 23 puede decirse lo mismo que sobre
    las 9,10. Aquí las consideraremos $\mathcal{O}(1)$
    \item [-] En la línea 18 se llama recursivamente al algoritmo. Por tanto:
    
    $$T(n)=\left\{\begin{array}{lr}
                   \frac{n^2-1}{2}\cdot T(n-1) & n>2\\
                   \ \\
                   1 & n=1
                  \end{array}\right.$$
                  
    y se tiene que:
    
    \begin{eqnarray} \label{efic}
    T(n)=\frac{n^2-1}{2}\cdot T(n-1)= \frac{n^2-1}{2}\cdot\frac{(n-1)^2-1}{2}
    \cdot T(n-2)=\\ \nonumber
    =\ldots=T(n-j-1)\cdot\frac{1}{2^{j+1}}\prod_{i=0}^j[(n-i)^2-1]
    \end{eqnarray}
    
    Tomando $j=n-3$ en (\ref{efic}), se tiene $T(n)=
    \mathcal{O}\left(\frac{(n!)^2}{2^n}\right)$
   \end{itemize}

   % Normalización e impresión de operaciones
   \newpage
   Puesto que en la lista de operaciones aparecen todas las operaciones necesarias
   para llegar a \C++{meta} o a una aproximación a la misma, pero también pueden
   más operaciones que las estrictamente necesarias, se presenta a continuación,
   otro algoritmo para normalizar dichas operaciones en función del resultado obtenido
   por el algoritmo \ref{Cifras} \\
   
   \begin{algorithm}[H]
   \begin{algorithmic}[1]
   
   	\IF{hay más de una \C++{Cuenta} en la lista de operaciones}
		\STATE{Llama al siguiente algoritmo, pasándole \C++{primero}
		y \C++{segundo} de la última \C++{Cuenta} efectuada, y como
		posición de escritura la penúltima de \C++{operaciones} (podría ser \C++{-1})}
	\ENDIF
	\REQUIRE \ \\
	    \C++{unaCuenta}, última cuenta necesaria en la lista\\
	    \C++{pos\_escribir}, posición anterior a la última normalizada\\
	
	\STATE{La \C++{Cuenta} a consultar es la que ocupa \C++{pos\_escribir}}
   % Falta comentar el caso base de la inducción
	\WHILE {No se hallen \C++{primero} y \C++{segundo} de \C++{unaCuenta}
   como resultado de otra \C++{Cuenta}, y quede alguna por consultar}
		\STATE{
			\IF {el resultado de la \C++{Cuenta} consultada es \C++{primero} o \C++{segundo}}
				\STATE{Marcarlo como encontrado}
				\STATE{Intercambiar la \C++{Cuenta} que ocupa la posición \C++{pos\_escribir} en 
				\C++{operaciones} por la \C++{Cuenta} consultada}
				\STATE{Decrementa \C++{pos\_escribir} y llama al algoritmo para la última
				\C++{Cuenta} consultada, y \C++{pos\_escribir} }
				\STATE{El índice a consultar es ahora \C++{pos\_escribir}}
			\ELSE
				\STATE{Decrementa el índice de la posición a consultar}
			\ENDIF
		}
	\ENDWHILE
   \end{algorithmic}
   \ \\
   Una vez normalizadas las operaciones:
   \begin{algorithmic}[1]
    \STATE{Se itera \C++{operaciones} desde el principio
    hasta el final de la lista}
    \PRINT{\C++{Cuenta} actual}
   \end{algorithmic}

   \caption{ALGORITMO DE NORMALIZACIÓN DE OPERACIONES}
   \label{Numeros}
   \end{algorithm}
   % Falta comentar cómo las operaciones para la mejor operación se copian a cada paso
\end{document}